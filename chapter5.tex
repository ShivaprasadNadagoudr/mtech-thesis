\chapter{Conclusions}
\label{ch:conclusions}

We formulated the problem of distributing the safe Bayesian Optimization by successively dividing the search space to create hyperspaces. We defined \emph{DistributedSafeOpt} and \emph{OverlappedDistributedSafeOpt} algorithms as a solution to the above-defined problem. We evaluated the proposed algorithms' performance via some standard test functions simulation and compared them with the baseline \emph{SafeOpt} algorithm.
We observed that distributed version algorithms reach the same or higher maxima as \emph{SafeOpt} within the same evaluation time constraint without evaluating more number unsafe evaluations. Also, communicating the points evaluated in the shared region in the \emph{OverlappedDistributedSafeOpt} algorithm helped reduce the unsafe evaluations compared to \emph{DistributedSafeOpt}. Furthermore, the distributed setup enforces some extra communication overhead, which can be accommodated as it takes significantly less time than the objective function evaluation.

We demonstrated the practical application of our algorithms in problem hyperparameter tuning of the CNN network and tuning of control parameters for a robotic arm. Training deep neural networks and running simulations for control problems demand high computing resources, so sequentially evaluating these problems for a set of hyperparameters requires more time to find maxima. So distributing the computation among all available nodes will reduce the time needed to find maxima. 

\section{Future Work}
\begin{itemize}
	\item We can create more copies of the hyperspaces that seem more promising in finding maxima if resources are available for use.
	\item We can early stop a hyperspace that is evaluating more unsafe points.
\end{itemize}