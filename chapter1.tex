\chapter{Introduction}\label{ch:introduction}
Reinforcement learning (Sutton and Barto. 1998 \cite{Sutton1998}) is the problem faced by an agent who must learn the behavior through trial-and-error interactions with a dynamic environment (Kaelbling et al. 1996 \cite{Littman1996}).
For example, AlphaZero (Silver et al. 2017 \cite{silver2017mastering}), developed by DeepMind, reached a virtually unthinkable level of play to defeat humans, and powerful chess game-engine Stockfish and reinforcement learning methods can play complex video games like Atari (Mnih et al. 2015 \cite{mnih2015humanlevel}) just by perception. 
This kind of performance is achieved by setting up the model with good hyperparameters and that model interacting with the environment a hundred or million times. 
The loss function of these models is not known as a function of hyperparameters. 
Non-availability of gradients of these functions makes the optimization process hard, and search methods are used for optimization. 
Bayesian optimization is one high sample efficient search method and was used as a routine service to adjust the hyperparameters of AlphaGo during its design and development cycle, resulting in progressively stronger agents (Yutian Chen et al. 2018 \cite{chen2018bayesian}).
Simulations are used to achieve this data-driven learning. 
So deploying these solutions, which pose safety constraints to both learner and environment, into the safety-critical environment is impossible.

Real consequences for actions in the real world force us to guarantee safety for real-world systems like robotics and nuclear power plant controller. 
For example, Turchetta et al.\cite{TurchettaB016} explored the challenge of a self-driving rover investigating Mars' surface.
The rover does not know the height or gradient of the surface it explores. 
Because the rover's ability to move across slopes is limited, it must study the surface safely in order to get the most scientific understanding. 
One seeks to avoid controls to rover causing physical harm to the platform in such situations. Such controls are obtained from hyperparameters to the control model; therefore, there is a safety constraint on hyperparameters. This is addressed by constrained or safe optimization.

%The agent in single-agent system models itself, the environment, and its interactions. 
%Multi-agent systems differ from single-agent systems in that several agents exist which model each other's goals and actions \cite{mas.survey}. 
%Since the multi-agent systems are intrinsically non-stationary, safety is a critical issue. 
The above example of the Mars rover uses SafeOpt \cite{SuiGBK15}, based on the Bayesian Optimization framework, to tune the model hyperparameters to achieve safety. 
For high dimensional functions, BO requires many more queries of the objective function to bound uncertainty about the optimization process.
So, we are parallelizing BO to evaluate more queries of the objective function within a given time constraint.
Prior work Hyperspace \cite{YoungHRK18} a parallel implementation of Bayesian Optimization, leverages high-performance computing (HPC) resources better to understand unknown, potentially non-convex hyperparameter search spaces. 
They show that it is possible to discover families of good hyperparameter settings over various Deep RL models by partitioning large search spaces and running many optimization procedures in parallel. 
Our work considers the successive partitioning of search space following safety constraints and runs the optimization process in parallel accordingly.

\section{Prior Work}
\label{sec:prior-work}
Multi-Armed (or $K$-armed) bandit problem models a situation where an agent faces a sequential series of $K$ possible choices, each leading to a random reward whose distribution is unknown. 
The goal of this agent is to maximize the total reward it receives over a given period. 
This is an Explore-Exploit dilemma, and we frequently face choosing between options. The \textit{UCB1} \cite{ucb1} (for Upper Confidence Bound) is an algorithm for the multi-armed bandit that achieves regret that grows only logarithmically with the number of actions taken.
Instead of $K$, discrete number arms, if we consider the case of an infinite number of arms in continuous space, will get continuous MAB called Bayesian Optimization \cite{Srinivas-Bandits}.

BO is an effective strategy for finding the extrema of objective functions that are costly to evaluate. 
Moreover, it is applicable in cases where one does not apriori know the objective function but can obtain evaluations (possibly noisy) of this function at sampled values. 
It is advantageous when these evaluations are costly, when the problem at hand is non-convex, or when one does not have access to find derivatives of the objective function \cite{Freitas-BO}.

Shahin et al. \cite{Shahrampour} addresses the multi-armed bandit problem in a multi-agent framework. 
Players explore a finite set of arms with stochastic rewards. 
The goal is to find the best global arm, i.e., the one with the largest expected reward averaged out among players. 
To achieve this goal, they developed \textit{Distributed Upper Estimated Reward} (\texttt{d-UER}), a distributed variant of the \textit{UCB1} algorithm. 
The algorithm performance is measured in terms of network regret. Complete network scales down the regret of its single-player counterpart by the network size.
Sankararaman et al. \cite{Sankararaman.ma.mab} consider a setup consisting of many agents $n$ that collaboratively and simultaneously solve the same instance of $K$-armed MAB to minimize the average cumulative regret over all agents. 
The agents can communicate and collaborate only through a pairwise asynchronous gossip-based protocol that exchanges a limited number of bits. The playing arms-set is non-decreasing. 
The local-explore with gossip mechanism ensures that the best arm spreads quickly through the network to all agents.
Chawla, Sankararaman et al. \cite{chawla2020gossiping} extend their previous work in the same setting, where gossipped best arm will be inserted into playing arms-set, and poor performing arm will be eliminated from playing arms-set. 
Agents pull information from other agents for a given communication budget.
Daniel Vial et al. \cite{Daniel.robust.ma.mab} assume the presence of malicious agents in the system. So they developed a \textit{blocking algorithm} that blocks an agent recommending poor arm.

Kandasamy et al. \cite{pmlr-v84-kandasamy18a} parallelized Bayesian Optimisation via Thompson Sampling and showed that making $n$ evaluations distributed among $M$ workers are essentially equivalent to performing $n$ evaluations in sequence. 
They developed synchronous and asynchronous parallel versions of Thompson Sampling (TS), which are synTS and asyTS, respectively.
Ruben et al. \cite{Ruben.distr.bo} compared Thompson sampling with other acquisition functions in a sequential setting, and it performed suboptimally. So they modeled BO as a partially observable MDP. They used Boltzmann policy in the query point sampling process, which trivially allows a distributed Bayesian optimization implementation with a high level of parallelism and scalability.

Extending and adapting deep\textcolor{white}{"}learning techniques for the sequential decision-making process, i.e., deciding based on recent experience a set of actions to take in an uncertain environment based on some goals, led to the development of deep reinforcement learning (DRL) approaches. 
Young Todd et al. \cite{YoungHRK18} address the challenge of hyperparameter optimization for DRL algorithms. They designed a distributed, sequential model-based optimization (SMBO) algorithm, \texttt{HyperSpace}, which achieves parallelism by concurrently running many Gaussian processes over a large hyperparameter search space. 
This is done by dividing the hyperparameter search space and forming hyperspaces. 
More critically, dividing the search space means that a more detailed search is performed as many more models are explored across a diverse set of search spaces. 
This addresses concerns of exponential scaling acknowledged in early papers on Bayesian optimization theory \cite{pmlr-v9-grunewalder10a} \cite{Srinivas.2012}, which showed that as number of search dimensions increase, Bayesian optimization requires many more queries of the objective function in order to bound our uncertainty about the optimization process.\textcolor{white}{"}

Garcıa and Fernández \cite{garcia15a} provide a thorough survey on Safe Reinforcement Learning. 
In many applications, we require that the sampled function values exceed some prespecified safety threshold, a requirement that existing algorithms fail to meet. 
Examples include medical applications where patient comfort must be guaranteed, recommender systems aiming to avoid user dissatisfaction, and robotic control, which seeks to avoid controls causing physical harm to the platform. 
Under the assumption that these problems (unknown function) satisfy regularity conditions expressed via a Gaussian process prior. 
Sui, Krause, et al.\cite{SuiGBK15} propose a novel algorithm, \texttt{SAFEOPT}, which models the unknown function as a sample from a Gaussian process (GP), and uses the predictive uncertainty to guide exploration.

\section{Our Work}
\label{sec:our-contribution}
\begin{itemize}
	\item We consider distributing the safe optimization process among multiple computing nodes by successively partitioning the search space to create new hyperspaces and deploying each hyperspace as a separate optimization process into a computing node.
	\item We propose the DistributedSafeOpt algorithm, which divides the hyperspaces during the expansion of the safe set.
	\item We consider the case overlapped hyperspaces and share the points evaluated in the shared region between computing nodes as an extension of the above algorithm called OverlappedDistributedSafeOpt.
	\item We are comparing the performance of our algorithms with SafeOpt as a baseline.
	\item We test our algorithms on some standard optimization test functions, an ensemble of functions for comparison.
	\item For demonstrating the applicability of our algorithms we consider two hyperparameter tuning problems - CNN and robot arm.
\end{itemize}


\section{Thesis Outline}
\begin{itemize}
	\item \textbf{Chapter } \ref{ch:background} This chapter introduced some necessary mathematical background and definitions related to Multi-armed bandits, Bayesian Optimization (BO) with Gaussian Processes and Acquisition functions, Safe BO, and parallel BO.
	\item \textbf{Chapter } \ref{ch:distbo} This chapter states the system model and problem statement for the distributed versions of safe BO. Explained the proposed algorithms with software implementation and an example showing the creation of hyperspaces
	\item \textbf{Chapter } \ref{ch:results} This chapter contains the performance analysis and comparison of all three algorithms for some standard optimization test functions and an ensemble of GP sample functions. Also, we discussed the practical applications like hyperparameter tuning of CNN network and tuning of control parameters of a Robot arm.
	\item \textbf{Chapter } \ref{ch:conclusions} This chapter concludes all the results and observations. Furthermore, we also discuss the future scope of this work.
\end{itemize}
