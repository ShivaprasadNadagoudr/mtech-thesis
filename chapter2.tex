\chapter{Background and Notation}\label{ch:background}
\section{The Explore-Exploit Tradeoff}
The real-world decision-making processes has a feature that presents uncertainity 
in the outcome of future decision. 
So it is difficult for agent(decision-maker), due to uncertainity in the result of a given action, to choose the best next action. 
Most of the real-world tasks aiming at maximising cumulative outcomes require to make many sequential decisions.\\
Successful completion of such a task necessitates a basic tension: an decision-maker (agent) must constantly choose between exploiting all 
known good possibilities and researching unknown but potentially better options. 
This conflict is known as the explore-exploit trade-off, and it's at the basis of improving decision-making.\\
The explore-exploit tradeoff can be observed in a variety of natural and artificial systems. 
Foraging animals in the natural world strive to consume as much food as possible while also seeking out the most rewarding foraging areas \cite{Keasar:2002}.

\section{Multi-Armed Bandit Problem}
The Multi-armed Bandit (MAB) problem is a classic mathematical description 
of the explore-exploit tradeoff \cite{Robbins:1952}. 
A decision-maker(agent) is faced with a sequential series of decisions in the MAB problem. 
Each choice requires the decision-maker(agent) to pick between two or more options, often known as arms, 
each of which has a probability distribution associated with it that models the reward. 
The decision-maker receives a noisy reward chosen from the related probability distribution after selecting an option. 
The goal of the decision-maker is to maximise their expected cumulative reward, 
which is comparable to selecting the option with the highest mean as often as possible.

\begin{algorithm}
	\SetAlgoVlined
	\caption{Distributed SafeOpt}
	\KwIn{Objective function $f$, GP prior($\mu, k$), Parameter search-space $S$, Number of parameters $d$, Number of subspaces per parameter $n$, Safe seed set $S_0$, Safety threshold $h$}
	\ForEach{$parameter$ in $S$}
	{
		divide $parameter$ into $n$ subspaces.
	}
	Take all combinations of subspaces to form $hyperspaces$.\\
	\tcc{$n^d$ hyperspaces are possible.}
	currentSafeHyperspace = $hyperspace : allHyperspaces | S_0 \in hyperspace$\\
	Deploy optimization process for whole search-space into a single node.
\end{algorithm}

\begin{algorithm}
	\SetAlgoVlined
	\caption{Deploy Hyperspace}
	\KwIn{$f$, GP($\mu, k$), $S$, $S_0$, $h$, $currentSafeHyperspace$, $allHyperspaces$}
	Initialize $SafeOpt$ with safe seed points $S_0$.\\
	\If{currentSafeHyperspace $is$ leafHyperspace}
	{
		\For{$i=1:10$}
		{
			run $SafeOpt$ algorithm
		}
		\KwRet{}
	}
	\For{$i=1:10$}
	{
		samplePoint=$SafeOpt$.optimize()\\
		funcValue=$f$(samplePoint)\\
		newSafeHyperspace = hyperspace : allHyperspaces | samplePoint $\in$ hyperspace\\
		\eIf{funcValue $\ge h$ AND newSafeHyperspace $\neq$ currentSafeHyperspace}
		{
			Split the search space between two hyperspaces.\\
			Deploy the newSafeHyperspace into new node with samplePoint as safe seed.\\
			Change the search space for currentSafeHyperspace accordingly and continue optimization process in same node.\\
		}
		{
			add samplePoint and funcValue to $SafeOpt$ model.
		}
		
	}
\end{algorithm}